\documentclass{article}
\usepackage{graphicx} 
\usepackage{xeCJK}
\usepackage{color}

\title{\textbf{\huge 系统工具开发基础(第一周实验报告)}}
\author{\LARGE 耿壮}
\date{August 2025}

\begin{document}
\begin{titlepage}
\maketitle
\end{titlepage}
\section{\LARGE 练习内容}
\subsection{Latex分部分}
\textbackslash{section}可以分好多大部分
\textbackslash{subsection}在section的基础上分的更细致
\subsection{Latex创建标题页}
\begin{flushleft}
\textbackslash{begin{titlepage}}\par
\textbackslash{maketitle}\par
\textbackslash{end{titlepage}}\par
为创建标题页
\end{flushleft}
\subsection{Latex添加摘要}
\begin{flushleft}
\textbackslash{begin{abstract}}  \par
\textbackslash{end{abstract}}\par
为添加摘要
\end{flushleft}
\subsection{Latex使用中文}
添加\textbackslash{usepackage\{xeCJK\}}和改为XeLaTex 编译器才能使用中文
\subsection{Latex换行的五种办法}
\begin{enumerate}    
\item 文本到下一段第一种方法是添加两个回车\newline
\item 文本到下一段第二种方法是在后面加“\textbackslash{par}”\newline
\item 文本到下一段第三种方法是在后面加“\textbackslash\textbackslash”\newline
\item 文本到下一段第四种方法是在后面加“\textbackslash{newline}”,但段首不缩进\newline
\item 文本到下一段第五种方法是在后面加“\textbackslash{hfill}+\textbackslash{break}”,但段首不缩进
\end{enumerate}
\subsection{Latex内容换页}
“\textbackslash{newpage}”前面加斜线后面的内容自动变到下一页
\subsection{Latex内容对齐}
\textbackslash{begin\{flushleft\}}\newline
\textbackslash{end\{flushleft\}}\newline
为放在两者内的内容变为左对齐\newline
\textbackslash{begin\{flushright\}}\newline
\textbackslash{end\{flushright\}}\newline
为放在两者内的内容变为右对齐
\subsection{Latex字体形式}
\begin{enumerate}
\item \textbackslash{textbf\{\}}为加粗\newline
\item \textbackslash{underline\{\}}为加下划线\newline
\item \textbackslash{textit\{\}}为变斜体\newline
\item \textbackslash{textit\{\textbackslash{textbf}\{\}\}}为斜体加粗复合
\end{enumerate}
\subsection{Latex改变字体Method1大小(由小到大排列)}
\begin{enumerate}
\item \{\textbackslash{tiny Method1}\}\newline
\item \{\textbackslash{scriptsize Method1}\}\newline
\item \{\textbackslash{footnotesize Method1}\}\newline
\item \{\textbackslash{small Method1}\}\newline
\item \{\textbackslash{normalsize Method1}\}\newline
\item \{\textbackslash{large Method1}\}\newline
\item \{\textbackslash{Large Method1}\}\newline
\item \{\textbackslash{LARGE Method1}\}\newline
\item \{\textbackslash{huge Method1}\}\newline
\item \{\textbackslash{Huge Method1}\}
\end{enumerate}
\subsection{Latex插入图片}
\textbackslash{usepackage\{graphicx\}}\newline
\textbackslash{graphicspath\{ \{./images/\} \}}\newline
创建名为images的文件夹上传图片名为universe的图片\newline
在插入图片的地方写\textbackslash{includegraphics\{universe\}}以插入图片
\subsection{Latex生成表格}
\begin{enumerate}
\item l 表示一个左对齐的列;\newline
\item r 表示一个右对齐的列;\newline
\item c 表示一个向中对齐的列;\newline
\item | 表示一个列的竖线;\newline
\item \& 用于分割列;\newline
\item \textbackslash\textbackslash 用于换行;\newline
\item \textbackslash{hline} 表示插入一个贯穿所有列的横着的分割线;\newline
\item \textbackslash{cline\{1-2\}} 会在第一列和第二列插入一个横着的分割线。\newline
\item 开始时用\textbackslash{begin\{tabular\}}\newline
 结束时用\textbackslash{end\{tabular\}}
\end{enumerate}
\subsection{Latex插入公式}
\begin{enumerate}
\item 开始时用\textbackslash{begin\{equation\}}\newline
 结束时用\textbackslash{end\{equation\}}\newline
\item 若碰到a,b,c等需加\&\newline
\item \$底数\^{}指数\$为上标\newline
\item \$底数\_指数\$为下标\newline
\item \$\$\textbackslash{frac\{a\}\{3\}}\$\$为三分之a(可嵌套)\newline
\item \$\$\textbackslash{sqrt[x]\{y\^{}2\}}\$\$为x次根号下y的平方\newline
\item \$\$\textbackslash{}sum\_\{x=1\}\^{}5 y\^{}z\$\$为x从一到五y的z次方求和\newline
\item \$\$\textbackslash{}int\_a\^{}b f(x)\$\$为从a到b f(x)的积分
\end{enumerate}
\subsection{git创建文件 }
touch file.txt为创建一个名为file的文件
\subsection{git查看日期 }
输入date
\subsection{git打印}
输入echo后加要打印的东西
\subsection{git 搜索}
\begin{enumerate}
\item in:name,description,readme spring为查找名称,描述或README文件中包含spring的文件按名称,描述或README搜索\newline
\item stars:>1000为查找星标超过1000的仓库按星标数量搜索\newline
\item forks:>500为查找复刻超过500次的仓库按复刻数量搜索\newline
\item language:python为查找主要使用python语言的仓库按编程语言搜索\newline
\item pushed:>2025-01-01为查找在2025年1月1日之后有更新的仓库按更新时间搜索\newline
\end{enumerate}
\subsection{git 在 shell 中导航}
当前工作目录可以使用 pwd 命令来获取\newline
切换目录需要使用 cd 命令
\subsection{git 输入输出}
echo hello > hello.txt将hello输入hello.txt\newline
cat < hello.txt输出hello.txt.内的东西
\subsection{git vim}
\begin{enumerate}
\item :q 退出(关闭窗口)
\item :w 保存(写)
\item :wq 保存然后退出
\item :e {文件名} 打开要编辑的文件
\item  :ls 显示打开的缓存
\item :help {标题} 打开帮助文档
\item :help :w 打开 :w 命令的帮助文档
\item :help w 打开 w 移动的帮助文档
\end{enumerate}
\subsection{git vim编辑}
\begin{enumerate}
\item i 进入插入模式
但是对于操纵/编辑文本,不单想用退格键完成
\item O / o 在之上/之下插入行
\item d{移动命令} 删除 {移动命令}
例如,dw 删除词, d\$ 删除到行尾, d0 删除到行头。
\item c{移动命令} 改变 {移动命令}
\item x 删除字符(等同于 dl)
\item s 替换字符(等同于 xi)
\item u 撤销, <C-r> 重做
\item y 复制 / “yank” (其他一些命令比如 d 也会复制)
\item p 粘贴
\end{enumerate}
\section{ \LARGE 结果与感悟}
学会了许多Latex和git的基础知识,对Latexh和git产生了浓厚兴趣,更加期待接下来的学习。
\section{ \LARGE github链接}
https://github.com/xingeng1/geng


\end{document}
